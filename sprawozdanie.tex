\documentclass{article}

\usepackage{polski}
\usepackage{amsmath,amsfonts,stmaryrd,amssymb} 
\usepackage[framemethod=tikz]{mdframed}
\usepackage[utf8]{inputenc}
\usepackage{geometry} 
\usepackage{titlesec}

\geometry{
	paper=a4paper,
	top=2.5cm, 
	bottom=3cm,
	left=2.5cm,
	right=2.5cm,
	headheight=14pt,
	footskip=1.5cm,
	headsep=1.2cm,
}

\titlelabel{\thetitle.\quad}

\pagenumbering{gobble}

\title{
	\textbf{Pracownia z analizy numerycznej}\\[8pt]
	\large{Sprawozdanie do zadania \textbf{P2.19}\\
	Prowadzący: dr hab. prof. Paweł Woźny}}

\author{\large{Martyna Firgolska, Michał Dymowski}}

\date{\large{Wrocław, \today}}

\newtheorem{theorem}{zera funkcji ortogonalnych}



\begin{document}

\maketitle 

\section*{Kwadratury Gaussa-Legendre'a}

\subsection*{Kwadratura} 
Metoda całkowania numerycznego polegająca na obliczeniu wartości wyrażenia
\[ \sum_{i=0}^n A_if(x_i) \approx \int_a^b f(x) d x \] 
gdzie $x_0,...,x_n\in [a,b]$.

\subsection*{Kwadratura Gaussa-Legendre'a}
Kwadratura w której węzły $x_0,x_1,...,x_n$ są pierwiastkami $n+1$-szego
wielomianu ortogonalnego na przedziale $[a,b]$, a współczynniki są równe
\[A_i=\int_a^b \prod_{j=0,j\neq i}^n \frac{x-x_j}{x_i-x_j} dx\]. O istnieniu
wystarczająco wielu pierwiastków wielomianu ortogonalnego, oraz zasadności 
takiego doboru węzłów mówią następujące twierdzenia: 

\begin{theorem}
	elo
\end{theorem}

\end{document}
